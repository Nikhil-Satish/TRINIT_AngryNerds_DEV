\documentclass[10pt]{article}
\usepackage[utf8]{inputenc}
\usepackage[T1]{fontenc}
\usepackage{amsmath}
\usepackage{amsfonts}
\usepackage{amssymb}
\usepackage[version=4]{mhchem}
\usepackage{stmaryrd}

\title{MATHEMATICS }

\author{}
\date{}


\begin{document}
\maketitle
\begin{enumerate}
  \item Find number of common terms in the two given series
\end{enumerate}

$4,9,14,19 \ldots \ldots$. up to 25 terms and

3, 9, 15, $21 \ldots$...up to 37 terms\\
(1) 4\\
(2) 7\\
(3) 5\\
(4) 3

Ans. (1)

Sol. $\quad 4,9,14,19, \ldots \ldots . .124 \rightarrow d_{1}=5$

$3,9,15,21 \ldots \ldots . .219 \rightarrow d_{2}=6$

$1^{\text {st }}$ common term $=9$ and common difference of common terms $=30$

Common terms are 9, 39, 69, 99

4 common terms

\begin{enumerate}
  \setcounter{enumi}{1}
  \item Let $8=3+\frac{3+p}{4}+\frac{3+2 p}{4^{2}}+\ldots \infty$, then $\mathrm{p}$ is\\
(1) 9\\
(2) $\frac{5}{4}$\\
(3) 3\\
(4) 1
\end{enumerate}

Ans. (1)

Sol. $8=3+\frac{3+p}{4}+\frac{3+2 p}{4^{2}}+$

multiply both sides by $\frac{1}{4}$, we get

$2=\frac{3}{4}+\frac{3+\mathrm{p}}{4^{2}}+$

Equation (i) - equation (ii)

$\Rightarrow 6=3+\frac{\mathrm{p}}{4}+\frac{\mathrm{p}}{4^{2}}+\ldots \ldots$

$\Rightarrow 3=\frac{\mathrm{p}}{4\left(1-\frac{1}{4}\right)} \Rightarrow \mathrm{p}=9$

\begin{enumerate}
  \setcounter{enumi}{2}
  \item For $\frac{\mathrm{x}^{2}}{25}+\frac{\mathrm{y}^{2}}{16}=1$, find the length of chord whose mid point is $\mathrm{P}\left(1, \frac{2}{5}\right)$\\
(1) $\frac{\sqrt{1681}}{5}$\\
(2) $\frac{\sqrt{1481}}{5}$\\
(3) $\frac{\sqrt{1781}}{5}$\\
(4) $\frac{\sqrt{1691}}{5}$
\end{enumerate}

Ans. (4)

Sol. By $\mathrm{T}=\mathrm{S}_{1}$

$\Rightarrow \frac{\mathrm{x}}{25}+\frac{\mathrm{y}}{16}=\frac{1}{25}+\frac{4}{25} \cdot \frac{1}{16}$

$\Rightarrow \frac{\mathrm{x}}{25}+\frac{\mathrm{y}}{40}=\frac{4+1}{100}$

$\Rightarrow \frac{\mathrm{x}}{25}+\frac{\mathrm{y}}{40}=\frac{1}{20}$

$\Rightarrow 8 \mathrm{x}+5 \mathrm{y}=10$

$\Rightarrow \frac{\mathrm{x}^{2}}{25}+\left(\frac{10-8 \mathrm{x}}{5}\right)^{2} \frac{1}{16}=1$

$\Rightarrow \frac{x^{2}}{25}+\frac{4}{25}\left(\frac{5-4 x}{16}\right)^{2}=1$

$\Rightarrow \mathrm{x}^{2}+\frac{(5-4 \mathrm{x})^{2}}{4}=25$

$\Rightarrow 4 \mathrm{x}^{2}+(5-4 \mathrm{x})^{2}=100$

$\Rightarrow 20 \mathrm{x}^{2}-8 \mathrm{x}-15=0$

$\mathrm{x}_{1}+\mathrm{x}_{2}=2$

$\mathrm{x}_{1} \mathrm{x}_{2}=\frac{-15}{4}$

length of chord $=\left|\mathrm{x}_{1}-\mathrm{x}_{2}\right| \sqrt{1+\mathrm{m}^{2}}$

$=\frac{\sqrt{1691}}{5}$

\begin{enumerate}
  \setcounter{enumi}{3}
  \item If $f(x)=x^{3}+x^{2} f^{\prime}(1)+x f f^{\prime \prime}(2)+f$ "'(3), then find $^{\prime}(10)$.
\end{enumerate}

Ans. (202)

Sol. $\quad f^{\prime}(x)=3 x^{2}+2 x f^{\prime}(1)+f^{\prime}(2)$

$f^{\prime \prime}(x)=6 x+2 f^{\prime}(1)$

$f^{\prime \prime}(3)=6$

$f^{\prime}(1)=-5$

$f^{\prime \prime}(2)=2$

$\Rightarrow \mathrm{f}^{\prime}(10)=300+20(-5)+2$

$=202$

\begin{enumerate}
  \setcounter{enumi}{4}
  \item Let $\int_{0}^{1} \frac{d x}{\sqrt{x+3}+\sqrt{x+1}}=A+B \sqrt{2}+C \sqrt{3}$ then the value of $2 A+3 B+C$ is\\
(1) 3\\
(2) 4\\
(3) 5\\
(4) 6
\end{enumerate}

Ans. (1)

Sol. On rationalising

$\int_{0}^{1} \frac{(\sqrt{x+3}-\sqrt{x+1})}{2} d x$

$=\frac{2}{3.2}\left\{(\mathrm{x}+3)^{3 / 2}-(\mathrm{x}+1)^{3 / 2}\right\}_{0}^{1}$

$=\frac{1}{3}\{8-3 \sqrt{3}-(2 \sqrt{2}-1)\}$

$=\frac{1}{3}\{9-3 \sqrt{3}-2 \sqrt{2}\}$

$=\left(3-\sqrt{3}-\frac{2 \sqrt{2}}{3}\right): \mathrm{A}=3, \mathrm{~B}=-\frac{2}{3}, \mathrm{C}=-1$

$\therefore 2 \mathrm{~A}+3 \mathrm{~B}+\mathrm{C}=6-2-1=3$

\begin{enumerate}
  \setcounter{enumi}{5}
  \item If $|z-i|=|z-1|=|z+i|, z \in C$, then the numbers of $z$ satisfying the equation are\\
(1) 0\\
(2) 1\\
(3) 2\\
(4) 4
\end{enumerate}

Ans. (2)

Sol. $\mathrm{z}$ is equidistant from $1, \mathrm{i}, \&-\mathrm{i}$

only $\mathrm{z}=0$ is possible

$\therefore$ number of $\mathrm{z}$ equal to 1

\begin{enumerate}
  \setcounter{enumi}{6}
  \item If sum of coefficients in $\left(1-3 x+10 x^{2}\right)^{n}$ and $\left(1+x^{2}\right)^{n}$ is $A$ and $B$ respectively then\\
(1) $A^{3}=B$\\
(2) $A=B^{3}$\\
(3) $A=2 B$\\
(4) $A=B$
\end{enumerate}

Ans. (2)

Sol. $A=8^{n}$

$$
B=2^{n}
$$

(B) $\therefore A=B^{3}$

\begin{enumerate}
  \setcounter{enumi}{7}
  \item Let $a_{1}, a_{2}, \ldots, a_{10}$ are 10 observations such that $\sum_{i=1}^{10} a_{i}=50$ and $\sum_{i \neq j}^{10} a_{i} \cdot a_{j}=1100$, then their standard deviation will be\\
(1) $\sqrt{5}$\\
(2) $\sqrt{30}$\\
(3) $\sqrt{15}$\\
(4) $\sqrt{10}$
\end{enumerate}

Ans. (1)

Sol. $\quad\left(\mathrm{a}_{1}+\mathrm{a}_{2}+\ldots .+\mathrm{a}_{10}\right)^{2}=50^{2}$

$\Rightarrow \sum \mathrm{a}_{1}^{2}+2 \sum_{\mathrm{i} \neq \mathrm{j}} \mathrm{a}_{\mathrm{i}} \mathrm{a}_{\mathrm{j}}=2500$

$\Rightarrow \sum \mathrm{a}_{1}^{2}=300$

$\sigma^{2}=\frac{\sum \mathrm{a}_{\mathrm{i}}^{2}}{10}-\left(\frac{\sum \mathrm{a}_{\mathrm{i}}}{10}\right)^{2}$

$\Rightarrow \sigma^{2}=5 \Rightarrow$ S.D $=\sqrt{5}$

\begin{enumerate}
  \setcounter{enumi}{8}
  \item If $f(x)=\left[\begin{array}{ccc}\cos x & -\sin x & 0 \\ \sin x & \cos x & 0 \\ 0 & 0 & 1\end{array}\right]$ then
\end{enumerate}

Statement-1 : $f(-x)$ is inverse of $f(x)$

Statement-2 : $f(x+y)=f(x) f(y)$\\
(1) Both are true\\
(2) Both are false\\
(3) Only statement 1 is true\\
(4) Only statement 2 is true

Ans. (1)

Sol. $f(x) f(y)=\left[\begin{array}{ccc}\cos x & -\sin x & 0 \\ \sin x & \cos x & 0 \\ 0 & 0 & 1\end{array}\right]\left[\begin{array}{ccc}\cos y & -\sin y & 0 \\ \sin y & \cos y & 0 \\ 0 & 0 & 1\end{array}\right]$

$=\left[\begin{array}{ccc}\cos (x+y) & -\sin (x+y) & 0 \\ \sin (x+y) & \cos (x-y) & 0 \\ 0 & 0 & 1\end{array}\right]$

$=\mathrm{f}(\mathrm{x}+\mathrm{y})$

$\therefore \mathrm{f}(\mathrm{x}) \mathrm{f}(-\mathrm{x})=\mathrm{f}(0)$

$=\mathrm{I}$

\begin{enumerate}
  \setcounter{enumi}{9}
  \item If $a=\lim _{x \rightarrow 0} \frac{\sqrt{1+\sqrt{1+x^{4}}}-\sqrt{2}}{x^{4}}$ and $b=\lim _{x \rightarrow 0} \frac{\sin ^{2} x}{\sqrt{2}-\sqrt{1+\cos x}}$ find $a \cdot b^{3}$\\
(1) 16\\
(2) 32\\
(3) -16\\
(4) 48
\end{enumerate}

Ans. (2)

Sol. $\quad \mathrm{a}=\lim _{\mathrm{x} \rightarrow 0} \frac{\sqrt{1+\mathrm{x}^{4}}-1}{\mathrm{x}^{4}\left[\sqrt{1+\sqrt{1+\mathrm{x}^{4}}}+\sqrt{2}\right]}$

$=\lim _{x \rightarrow 0} \frac{x^{4}}{x^{4}\left[\sqrt{1+\sqrt{1+x^{4}}+\sqrt{2}}\right]\left[\sqrt{1+x^{4}}+1\right]}$

$=\frac{1}{2 \sqrt{2} \times 2}=\frac{1}{4 \sqrt{2}}$

$b=\lim _{x \rightarrow 0} \frac{\sin ^{2} x}{(1-\cos x)}(\sqrt{2}+\sqrt{1+\cos x})$

$=2 \times(\sqrt{2}+\sqrt{2})=4 \sqrt{2}$

$\therefore \mathrm{ab}^{3}=(4 \sqrt{2})^{2}=32$


\end{document}